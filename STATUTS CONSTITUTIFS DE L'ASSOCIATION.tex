\documentclass[a4paper]{article}
\usepackage[utf8]{inputenc}
\usepackage[T1]{fontenc}
\author{PEPIN Yoël et JADOUL Thibault}
\title{STATUTS CONSTITUTUFS DE L'ASSOCIATION}

\begin{document}



TERMINOLOGIE:
Association : désigne l'association constitué par le présents statuts.
Bureau : désigne le Président ainsi que les autre membres en charge de l'Association.

Les membres fondateur de l'Association ,dont la liste est annexée aux statuts constitutifs de l'Association, ont établi ainsi qu'il suit les statuts de l'association devant exister entre eux et toute autre personne qui viendrait ultérieurement à acquérir la qualité de membre de l'Association.

IL A ÉTÉ CONVENU ET DÉCIDE CE QUI SUIT :


ARTICLE 1 - DÉNOMINATION DE L'ASSOCIATION
Association a pour dénomination : WI-BASH.
Cette association est régie par la loi du 1er juillet 1901 et pas le décret du 16 août 1901.

ARTICLE 2 - OBJET
Cette Association a pour objet : de développer des projets informatiques et de faire découvrir les possibilité que nous offre ce domaine.

ARTICLE 3 - SIÈGE SOCIAL
Le siège social est fixé à l'adresse suivante : 86 Résidence La Désirade, 97139 - ABYMES
Il peut être transféré par décision de l'assemblé générale.

ARTICLE 4 - DURÉE
La durée de l'Association est illimitée.

ARTICLE 5 - RESSOURCES
Les ressources de l'association se composent :
- du montant des cotisation et des droits d'entrée.
- des éventuelles subventions de l'Etat et des collectivités territoriales.
- de toute autre ressource autorisé par la Loi.

ARTICLE 6 - MEMBRES - COTISATIONS
L'Association est composée de :
-membres fondateurs.
-membre actifs dit " adhérent".
-membre d'honneur.
-membre bienfaiteurs.

Les membres peuvent être des personnes physiques ou morales.
A l’exception des membre d'honneur, tous les membres ont le droit de vote dans les assemblées.
Sont membres adhérents les personnes au participent au fonctionnement de l' Association et à la réalisation de son objet.
Sont membres fondateur de l'Association les membres qui ont participé à  sa construction et dont la liste est annexée aux présents statuts.
Les membres adhérents et les membres fondateurs s'engagent à versée une cotisation annuelle à l’Association, dont le montant est fixée en assemblée générale.
Sont membre d'honneur les personnes ayant rendu des services particulier à l'Association. Il sont dispensés du paiement de la cotisation.
Ils sont désignés par le Bureau, à la majorité des voix des membres présents.
Sont membres bienfaiteurs les personnes qui soutiennent financièrement l’association et paiement une cotisation plus élevée que celle des adhérents.
Les membres de l'association ne sont pas personnellement responsable des engagements contractés par elle.


Article 9 - BUREAU
L'Association choisit parmi ses membres personnes physiques un Bureau composé de :
    -Le Président de l'Association qui pourra éventuellement être accompagné d'un ou plusieurs Vice-Présidents;
    -Un Trésorier, et, si besoin est, un Trésorier adjoint.*
Les fonctions de Président et de Trésorier ne sont pas cumulables.
Sont nommés membres du Bureau:
    -PEPIN Yoël, de nationalité Française domicilié au 86 Résidence La Désirade, 97139 Les Abymes, Etudiant à L'UA
    -JADOUL Thibault, de nationalité Belge domicilié à Maison Gyu Pierre, Chemin de Leroux , 97190 Le Gosier,Etudiant à L'UA.
9.1 - Président
Le Président assume les fonctions de représentation légale judiciaire et extrajudiciaire de l'Association dans tous les actes de la vie civile.

Il préside l'Assemblée Générale , le Conseil d'administration et le Bureau.

Il peut être assisté par un ou plusieurs Vice-Présidents qui le remplace(nt) en cas d'empêchement.

Le Président est élu pour une durée de 1 an.

Son mandat est renouvelable 3 fois .


9.2 - Trésorier
Le Trésorier veille à l'établissement des comptes annuels de l'Association.

Il tient une comptabilité régulière des opérations et rend comte de sa gestion chaque année devant l'Assemblée Générale.

Il est en charge de l'appel des cotisations, du paiement et de la réception de toutes sommes. Il établit un rapport sur la situation financière de l'Association, soumis à l'Assemblé Générale .

9.3 - Attributions du Bureau

Le Bureau assure la gestion courante de l'Association.

Article 10 - ASSEMBLÉE GÉNÉRALE ORDINAIRE

L'Assemblée générale se réunit au moins une fois par an.

Quinze(15) jours au moins avant la date fixée, les membres de l'Association sont convoqués par le Président.

La convocation se fait par tous les moyens.

L'ordre du jour figure sur les convocations

Le Président préside l'assemblé et expose la situation morale ou l'activité de l'Association. Le Trésorier rend compte de sa gestion et soumet les comptes annuels (bilan, compte de résultat et annexe) à l'approbation de l'assemblée.

L'assemblée délibère sur les orientations à venir.

Ne peuvent être abodés que les points inscrits à l'ordre du jour.

Les décisions sont prises à la majorité des voix des membres présents, chaque memebre disposant d'une voix à l'exception des membres d'honneurs .

Toutes les délibérations sont prises à bulletin secret et constatées sur un procès-verbal signé par le Président.

Article 10 - ASSEMBLÉE GÉNÉRALE EXTRAORDINAIRE

Si besoin est, ou sur la demande de la moitié plus un des membres inscrits, le Président peut convoquer une Assemblée générale extraordinaire, uniquement pour modification des statuts ou pour prononcer la dissolution de l'Association ou statuer sur des actes portant sur des immeubles.

Les modalités de convocation sont les mêmes que pour l'assemblée générale ordinaire.

Les décisions sont prises à la majorité des voix des membres présents.

Toutes les délibérations sont prises à bulletin secret.

Article 12 - INDEMNITÉS

Toutes les fonctions, y compris celles du Bureau, sont gratuites et bénévoles.

Les frais occasionnés par l'accomplissement de leur mandat sont remboursés sur justificatifs. Le rapport financier présenté à l'assemblée générale ordinaire présente, par bénéficiaire, les remboursements de frais de mission, de déplacement ou de représentation. Ces dispositions peuvent être affinées dans un règlement intérieur (nature des frais,qualité des bénéficiaires,etc.)

Article 13 - RÈGLEMENT INTÉRIEUR

Un règlemet intérieur peut être établi par l'assemblée générale.

Il s'imposeà tous les membres, au même titre que les statuts. Il précise les règles de fonctionnement et d'organisation de l'Association, ainsi que tous les éléments jugés utiles pour le bon fonctionnement de l'Association qui ne sont pas prévus dans les présents statuts.

Article 14 - DISSOLUTION

En cas de dissolution, un ou plusieurs liquidateurs sont nommés, et l'actif, s'il y a lieu, est dévolu conformément aux décisions de l'assemblée générale extraordinaire qui statue sur la dissolution.

Les membres de l'Association ne peuvent se voir attribuer, en dehors
\end{document}